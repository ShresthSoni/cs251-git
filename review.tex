\documentclass[a4paper,12pt]{article}
\usepackage[utf8]{inputenc}

\title{Sholay}
\author{Shaswat Kar }
\date{4 April 2017}

\begin{document}

\section{Overview}

No film before or after Sholay has achieved as much - running to full houses in an unbroken 6 year stretch! Consider the fact that this happened in India - where the audiences have the most choices amongst all the people of the world. Consider the fact that no movie anywhere in the world has been able to reap even half this success that Sholay enjoys.

\section{Speciality}

It is potent, does not pretend, and has clearly been made with courage and conviction. It was made for the enjoyment of its predominantly Indian, fun-loving audiences, so my dear detractors from the clan of the perennially cynical, reality-fed, fun deprived masses of the world, kindly excuse yourselves. Sholay is quintessential Mumbai produce - you either consume and enjoy in total acceptance, or you're not fit for the ride. It is as simple as that.

The world would be a poorer place without films like Sholay, not only because of the almost arrogant instantaneous and total suspension of disbelief it demands from its audiences, but because of the honesty and sheer power of that demand. You go to a boxing match to see two boxers fight it out, not to see them playing chess in the middle of the ring!

\section{Everlasting}

Sholay will always be an incredible and shining example of purity, because it was the purest attempt made to entertain, with absolutely no quarters even considered for critical acclaim from the so called pundits of cinema.

Yours truly is guilty of seeing this film 23 times to date, 17 of these in the theatres. And if opportunity allows, I wouldn't mind being sentenced to death with more of this guilt delightfully added to my conscience.

As with any other film, there have been many comments on what is wrong with Sholay. To these, here is my personal quote - "A good film is about getting most things right, not about getting the fewest things wrong!"

\section{Conclusion}

And here is my response to other heartless comments on Sholay being a copy of some other film or filmmaker's work - "If every human was an original, we would have 6 billion different species without a single one to call friend or family".

To argue against failure may be human, but to argue against success is sub-human. You may limit yourselves if you so wish.
\end{document}