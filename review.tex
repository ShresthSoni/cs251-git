\documentclass[a4paper,12pt]{article}
\usepackage[utf8]{inputenc}

\title{Sholay}
\author{Shaswat Kar }
\date{4 April 2017}

\begin{document}
<<<<<<< HEAD

\section{Overview}

No film before or after Sholay has achieved as much - running to full houses in an unbroken 6 year stretch! Consider the fact that this happened in India - where the audiences have the most choices amongst all kinds of the people of the world, in every part of the world. Consider the fact that no movie anywhere in the world has been able to reap even half this success that Sholay enjoys.

\section{Speciality}

It is potent, does not pretend, and has clearly been made with courage and conviction. It was made for the enjoyment of its predominantly Indian, fun-loving audiences, so my dear detractors from the clan of the perennially cynical, reality-fed, fun deprived masses of the world, kindly excuse yourselves. Sholay is quintessential Mumbai produce - you either consume and enjoy in total acceptance, or you're not fit for the ride. It is as simple as that.

The world would be a poorer place without films like Sholay, not only because of the almost arrogant instantaneous and total suspension of disbelief it demands from its audiences, but because of the honesty and sheer power of that demand. You go to a boxing match to see two boxers fight it out, not to see them playing chess in the middle of the ring!

\section{Everlasting}

Sholay will always be an incredible and shining example of purity, because it was the purest attempt made to entertain, with absolutely no quarters even considered for critical acclaim from the so called pundits of cinema.

Yours truly is guilty of seeing this film 23 times to date, 17 of these in the theatres. And if opportunity allows, I wouldn't mind being sentenced to death with more of this guilt delightfully added to my conscience.

As with any other film, there have been many comments on what is wrong with Sholay. To these, here is my personal quote - "A good film is about getting most things right, not about getting the fewest things wrong!"

\section{Conclusion}

And here is my response to other heartless comments on Sholay being a copy of some other film or filmmaker's work - "If every human was an original, we would have 6 billion different species without a single one to call friend or family".

To argue against failure may be human, but to argue against success is sub-human. You may limit yourselves if you so wish.
\end{document}
=======
\begin{center}
\textmd {\huge 3 IDIOTS}
\end{center}

\begin{flushleft}
\textbf{Story}\\
The film begins with Farhan(Madhavan) and Raju(Sharman Joshi) beginning a journey in search of their lost friend Rancho(Aamir Khan) who leaves college all of sudden. The three are roommates and go about their life at college in a lively way. Rancho used to guide his friends in understanding the real purpose of life and education. The villain of the flick whose ways are against that of Rancho is the Dean of the college Viru Sahastrabuddhe(Boman Irani). In addition, to make matters worse Rancho falls in love with his daughter Pia(Kareena Kapoor). Does Farhan and Raju found Rancho or not? Why did Rancho leave college all of a sudden forms the rest of the story.\\
\vspace{3mm}

\textbf{Performance}\\
The movie is filled with great performances from all the actors.
\vspace{3mm}

\textbf{Verdict}\\
3 idiots is the best film of the Year 2009. A must watch.
\vspace{3mm}

\begin{center}
\textmd {\huge SHOLAY}
\end{center}
\textbf{Story}\\
Notorious dacoit Gabbar escapes jail but he is enraged at the man who put him there in the first place - Thakur. As an act of malicious revenge, Gabbar and his henchmen kill Thakur´s entire family (except for his daughter-in-law). Now Thakur wants to bring Gabbar to justice. He is helpless because he lost his arms and he is out of the police force. Therefore, he employs the help of two brave but bad layabouts, Jai (Amitabh Bachchan) and Veeru (Dharmendra), to bring Gabbar to him. He wants a helpless Gabbar lying at his feet.\\
\vspace{3mm}

\textbf{Performance}\\
All the actors have done a great job in in their respective roles, especially Gabbar. Amjad Khan has been spine chilling as well as a treat to watch as Gabbar. The dialogues are the life of this movie. Some of them even remembered today. Direction of Ramesh Sippy should be credited for this once in a lifetime movie.\\ 
\vspace{3mm}

\textbf{Verdict}\\
This movie requires no verdict. The fact that this movie is still very much alive in the hearts of people says everything. A classic that just needs to be marvelled at. A once in a lifetime movie.\\
And this doesn't create conflict in the minds of people.\\
\end{flushleft}
\end{document}
>>>>>>> 071c875658605359be5ddba24d40aeb4b548f9b6
